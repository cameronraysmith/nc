\section{Project Summary}
\subsection{Intellectual Merit}
\noindent The proposed work addresses fundamental questions regarding the computational capabilities of the central nervous systems of living organisms. The associated questions are fundamental to an integrative understanding of structure-function physiology and co-evolution in all organisms with nervous systems and the evolutionary predecessors that gave rise to them. With respect to nervous system organization, this project will deepen previous explorations into the interaction of developmental and environmental constraints on the architecture and complementary computational capacity of networks of living neurons. On the theoretical side, focusing on flexible abstract models of computation, rather than those specific to digital electronics, will enable the investigation of potentially novel computational potential implemented via architectures that differ from digital electronics in their apparent ability to take advantage of synergy between stochastic and deterministic properties. On the experimental side, cell cultured neurons will be developmentally selected for their capacity to robustly perform computations supporting various logical frameworks and the associated network architectures will be reverse engineered for further study. This investigation may, therefore, in addition to contributing to fundamental biological understanding, identify computational principles unique to natural neural networks. This project will embody the value of linking theoretical, computational and experimental methods of inquiry. This is necessary to address questions regarding co-evolution and co-development between underlying biological organization and computational tasks such structures are capable of performing in a variety of environmental conditions.
\subsection{Broader impacts}
\noindent Virtual interaction among all involved, including the public, will be fostered by an open online \href{http://en.wikipedia.org/wiki/Wiki}{Wiki} that will be used to organize and collaborate on this project and, more generally, support the movement for \href{http://en.wikipedia.org/wiki/Open\_notebook\_science}{Open Notebook Science}. All computer code will be made open source in accordance with the \href{http://opensource.org/licenses/MIT}{MIT license} and the codebase history will be available to the public free and in real-time on \href{http://www.github.com}{github}. The Wiki will contain tutorials on important concepts relating to the development of this project that will be accessible to motivated high-school students and undergraduates who may be considering interdisciplinary training for future careers in science. The foundations of this research and results obtained throughout its development will be incorporated directly into graduate courses at both the U.S.-based and Israel-based PI’s respective institutions.

The Bergman and Moses labs both combine theory and experiment. They do so in different ways and exposure to each of these models will be crucial for the education of students and postdoctoral fellows as they move on to make important decisions in their careers. In the Bergman lab, three Ph.D. students will be involved in developing the theoretical aspect of this project in close interaction with the PI. In the Moses lab two students and a postdoctoral fellow will be involved. This culturally diverse and interdisciplinary team of experienced and developing scientists will enable the success of the research program. Extensive interaction among those involved in performing the research will be fostered by frequent video conferences and an exchange program that we plan to engage in at crucial theory-experiment integration stages throughout the project. This international collaboration will provide a foundation for future collaborations among the PIs, students and postdoctoral fellows involved.
\pagebreak