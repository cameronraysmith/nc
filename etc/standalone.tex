\documentclass[]{article}
\usepackage{amssymb,amsmath}
\usepackage{graphicx}
% We will generate all images so they have a width \maxwidth. This means
% that they will get their normal width if they fit onto the page, but
% are scaled down if they would overflow the margins.
\makeatletter
\def\maxwidth{\ifdim\Gin@nat@width>\linewidth\linewidth
\else\Gin@nat@width\fi}
\makeatother
\let\Oldincludegraphics\includegraphics
\renewcommand{\includegraphics}[1]{\Oldincludegraphics[width=\maxwidth]{#1}}
\usepackage[unicode=true,
              colorlinks=true,
              linkcolor=blue]{hyperref}
\hypersetup{breaklinks=true, pdfborder={0 0 0}}
\setlength{\parindent}{0pt}
\setlength{\parskip}{6pt plus 2pt minus 1pt}
\setlength{\emergencystretch}{3em}  % prevent overfull lines
\setcounter{secnumdepth}{0}
\usepackage{cite}

%------------------%
% http://yav.purely-functional.net/haskell_latex.html
\usepackage{fancyvrb}
\DefineVerbatimEnvironment{code}{Verbatim}{fontsize=\small}
\DefineVerbatimEnvironment{example}{Verbatim}{fontsize=\small}
\newcommand{\ignore}[1]{}
%------------------%

%------------------%
% http://en.wikibooks.org/wiki/LaTeX/Packages/Listings
\usepackage{listings}
\usepackage{color}

\definecolor{dkgreen}{rgb}{0,0.6,0}
\definecolor{gray}{rgb}{0.5,0.5,0.5}
\definecolor{mauve}{rgb}{0.58,0,0.82}
 
\lstset{ %
  language=Caml,                % the language of the code
  basicstyle=\footnotesize,           % the size of the fonts that are used for the code
  numbers=left,                   % where to put the line-numbers
  numberstyle=\tiny\color{gray},  % the style that is used for the line-numbers
  stepnumber=2,                   % the step between two line-numbers. If it's 1, each line 
                                  % will be numbered
  numbersep=5pt,                  % how far the line-numbers are from the code
  backgroundcolor=\color{white},      % choose the background color. You must add \usepackage{color}
  showspaces=false,               % show spaces adding particular underscores
  showstringspaces=false,         % underline spaces within strings
  showtabs=false,                 % show tabs within strings adding particular underscores
  frame=single,                   % adds a frame around the code
  rulecolor=\color{black},        % if not set, the frame-color may be changed on line-breaks within not-black text (e.g. commens (green here))
  tabsize=2,                      % sets default tabsize to 2 spaces
  captionpos=b,                   % sets the caption-position to bottom
  breaklines=true,                % sets automatic line breaking
  breakatwhitespace=false,        % sets if automatic breaks should only happen at whitespace
  title=\lstname,                   % show the filename of files included with \lstinputlisting;
                                  % also try caption instead of title
  keywordstyle=\color{blue},          % keyword style
  commentstyle=\color{dkgreen},       % comment style
  stringstyle=\color{mauve},         % string literal style
  escapeinside={\%*}{*)},            % if you want to add LaTeX within your code
  morekeywords={*,...}               % if you want to add more keywords to the set
}
%------------------%
\usepackage[toc]{multitoc}

\title{Computation in neural systems}

\begin{document}
\maketitle
\tableofcontents
\section{Motivating questions}

What are the structural properties at the level of networks of neurons,
modules of networks of neurons, and perhaps higher order forms of
organization necessary to support the capacity for
\href{http://en.wikipedia.org/wiki/Metalinguistic\_abstraction}{abstraction},
which is fundamental to computation and may likewise be fundamental to
\href{http://en.wikipedia.org/wiki/Cognitive\_architecture}{cognitive
architecture}, development, and function \cite{Tenenbaum2011}? Given that boolean
logic devices are capable of being implemented in vitro using
neurons\cite{Feinerman2008}, is it possible to biologically engineer analogous
devices capable of performing computations that have natural embeddings
within \href{http://en.wikipedia.org/wiki/First-order\_logic}{first} or
\href{http://en.wikipedia.org/wiki/Higher\_order\_logic}{higher order
logic}? Moreover, can we define environmental conditions that lead to
the development of an extensionally equivalent machine, but whose
architecture may differ from the human-engineered biological
implementation? Would multiple developmental implementations conserve
identifiable structural features? And, how do the structural properties
of networks with the capacity for zeroth, first, or higher order
computations compare?

\section{Physical implementations of abstractly characterized
computation}

Since the development of the electronic digital computer, which makes
use of the digital abstraction \cite{Ward1989} from analog electrical circuits
there has been a close heuristic association between boolean logic and
computation. However, developments in formal logic over the past century
have been largely motivated by its applications to computation and the
theory of programming languages that sometimes build upon and sometimes
go beyond boolean logic to support additional forms of abstraction. In
fact, the capacity to support abstraction mirroring various systems of
formal logic has become a means to order and thereby assign value to
programming languages \cite{Abelson1996}. Electrictronic devices that are capable
of supporting such forms of abstraction provide a concrete physical
instantiation of the ideas inherent to the logical systems they are
designed to faithfully implement.

Neuronal logic devices (NLDs) represent an alternative physical modality
to digital electronics for the purpose of performing computation.
However, rather than attempting to directly parallel the history of the
development of digital electronics via the digital abstraction,
high-level descriptions of computation, such as the
$\lambda$-calculus, serve as a specification
of computation that is agnostic to the physical modality of
implementation. If the specification that a neuronal computation device
should be capable of implementing the
$\lambda$-calculus, a natural first step
toward this broad goal is to investigate simple neuronal systems that
are capable of performing well-defined computations that require a
capacity for first- or higher-order logic.

\section{A higher-order function to be implemented as a higher-order
NLD}

The $\lambda$-calculus notation is helpful in
order to define any higher order function. Some of the notation may be
implicitly familiar to users of imperative programming languages under
the heading ``anonymous functions''. We provide only an extremely
informal set of examples necessary to explain our intended work;
however, complete details can be found in \cite{Barendregt1985}. We can define a
standard binary boolean function like the ``and'' function as
\begin{equation}\label{eq:bool}
\lambda x. \lambda y.(\wedge\,\,x\,\,y)
\end{equation}
Such a lambda expression can apparently be applied to any inputs;
however the inputs to such a function are not necessarily restricted to
booleans unless we infer that the standard logical operator
``$\wedge$'' only accepts boolean arguments.
Note that we have used the
\href{http://en.wikipedia.org/wiki/Polish\_notation}{prefix or Polish
notation} for the $\wedge$ operator, which
in the perhaps more common infix notation is written with its arguments
flanking the operator as $x \wedge y$ to mean
``$x$ and
$y$''. We can provide explicit type
annotations for the bound variables $x$
and $y$, which indicate the types of
the arguments
\begin{equation}\label{eq:booltype}
\lambda x:bool. \lambda y:bool.(\wedge\,\,x\,\,y)
\end{equation}
We can now also provide a type annotation for this expression as a whole
\begin{equation}\label{eq:boolfulltype}
[\lambda x:bool. \lambda y:bool.(\wedge\,\,x\,\,y)]:[bool \rightarrow bool \rightarrow bool]
\end{equation}
Depending upon conventions with respect to
\href{http://en.wikipedia.org/wiki/Currying}{currying}, one could read
the type annotation as ``the function that takes two arguments each of
type $bool$ and returns a values of type
$bool$'' or ``the function that takes an
argument of type $bool$ and returns a
function that takes an argument of type
$bool$ and returns a value of type
$bool$''. The first formulation may be
easier to read, but the second is standard as a result of the way in
which functional programming languages implement such functions.

Now we can imagine that if we wish to abstract from the particular
binary boolean function implied by the
$\wedge$ operator, we need to introduce a
functional variable, whose type will be explicitly denoted for
concreteness despite the fact that it could be inferred, for which this
operator can be substituted. Doing this results in the following second
order function
\begin{multline}\label{eq:lamhobf}
[\lambda f:(bool \rightarrow bool \rightarrow bool).
\lambda x:bool. \lambda y:bool.(f\,\,x\,\,y)]
\\
:[(bool \rightarrow bool \rightarrow bool) 
\rightarrow bool \rightarrow bool \rightarrow bool]
\end{multline}
This function is intended to be read, in the less verbose uncurried
form, as ``the function that takes as its first argument (a function
that takes two boolean values as arguments and returns a boolean value)
and as its second and third arguments two boolean values and returns a
boolean value''. It is perhaps remarkable that this simple abstraction
is now capable of implementing any of the 16 possible boolean functions,
provided that the proper binary boolean operator is submitted as the
first argument to this function. Furthermore, the statement of this
function in terms of a simply typed lambda calculus notation is agnostic
to any physical implementation capable of realizing extensionally
equivalent behavior.

To make this more concrete, we can very simply implement the above
function in any functional programming language, or any programming
language supporting the passing of function handles to other functions.
An implementation in the programming language OCaml appears as follows
\begin{lstlisting}[caption={a higher order boolean function},label=camlhobf]
let hobf = fun (bf : ('a 'a bool)) (i1 : bool) (i2 : bool)
	       -> bf i1 i2
\end{lstlisting}
What is required in order to evaluate this function are corresponding
implementations of boolean functions to be substituted either for
$f$ in the lambda calculus notation or
for bf in terms of the corresponding OCaml implementation. For example
we can implement the XOR function using pattern matching to define a
truth table.
\begin{lstlisting}[caption={implementation of an XOR boolean operator},label=camlXOR]
let xor p1 p2 = match (p1, p2)
with (false, false) -> false
   | (false,  true) -> true
   | ( true, false) -> true
   | ( true,  true) -> false
\end{lstlisting}
Other binary boolean functions are implemented in an analogous fashion.
In order to evaluate hobf we then simply provide the name of a binary
boolean function and two boolean values as
\begin{lstlisting}[caption={Example output of hobf},label=output]
>hobf xor 0 0 = 0
>hobf xor 0 1 = 1
>hobf xor 1 0 = 1
>hobf xor 1 1 = 0
\end{lstlisting}

\section{Assessment of abstraction potential in NLDs}

An important consideration is to state precisely some criterion for
determining that a particular NLD has achieved potential for an explicit
form of abstraction such as is indicated in the relationship between
expressions \ref{eq:boolfulltype} and
\ref{eq:lamhobf}. In abstracting the binary boolean
operator $\wedge$ to the functional variable
$f$, which is the fundamental
transformation enabling the derivation of
\ref{eq:lamhobf} from
\ref{eq:boolfulltype}, we imply that any physical
implementation must take at least three rather than two inputs and the
first of these must specify a particular binary boolean operator to
apply to the latter two boolean input values. This roughly means that
any system that at least partially implements the lambda expression or
function specified in equation \ref{eq:lamhobf} and
listing \ref{camlhobf} respectively , must be capable of
interpreting the concept of ``selection from a set'' whose size is
determined by the subset of the 16 binary boolean operators that is
already implemented in a lower-level form. Indeed another representation
of equation \ref{eq:lamhobf} as a partial or total
set function could be written as $hobf : Hex \times Bool \times Bool \rightarrow Bool$
where we interpret $Bool$ and
$Hex$ as two and sixteen element sets
respectively. This point of view makes clear that the capacity for
selection from two element sets is already apparent in the binary
boolean operator written as a set function
$\wedge : Bool \rightarrow Bool$. The difference between these is
that the system must implement the typing constraints necessary to
distinguish a set that takes on sixteen possible values from one that
takes on two. For example, in the application of the function
$hobf$ the following evaluate as expected
given the definition: $hobf \wedge \,\, 1\,\, 0=0$ or
$hobf \wedge \,\, 1\,\, 1=1$. However, what is to be expected
given inputs such as: $hobf 1 \,\, \wedge\,\, 0$ or
$hobf 1 \,\, 1\,\, \wedge$? In these cases an output type
intuitively associated to $error$ is
required to indicate that the realization of a system implementing
equation \ref{eq:lamhobf} has indeed correctly
implemented the necessary typing constraints to claim that the function
has been realized. In the case of neuronal logic devices, this
alternative output should differ in a measurable way from those
associated to $1$ and $0$.

\section{Utilizing synaptic plasticity to generate higher order NLDs}

In order to take advantage of the immense capacity for computation
available in biological systems and neural networks, the biological
complexity of each neuron and synaptic connection must be taken into
consideration. Namely, the presence of paired pulse depression and
facilitation, which has been previously demonstrated in cultured
hippocampal neurons \cite{Cummings1996}, could be exploited to create neural objects
with similar design, but more complex behavior than existing NLDs.
Specifically, if paired pulse coordination can be shown to exist,
especially in the forward direction more significantly than the reverse,
then perhaps chains of action potentials could be used as the logical
readouts instead of single action potentials. If this is the case, then
something approaching a hexadecimal computing system can be created
using these logical elements, which would allow for significant
information compression. In the long term, something approximating the
design of a ``neural VLSI'' supporting a full logic system such as the
simply typed lambda calculus could be implemented.

\section{From NLDs to principles of cognition}

From the point of view that identifies the capacity for abstraction with
computation, the investigation of neuronal logic devices capable of such
function provides a framework in which various hypotheses from cognitive
science could begin to be evaluated at the level of well-defined neural
circuits. By attempting to isolate minimal implementation criteria, this
approach may serve to complement, enable simpler explanations of, or
identify paradoxical results derived from studies that treat whole
brains and their associated sensory apparatus as their object of study
\cite{McClelland2010,Griffiths2010}.

\bibliographystyle{unsrt}
\bibliography{/home/cameron/Downloads/bibtex/hierarchicalorganization,/home/cameron/Downloads/bibtex/books}

%1. Tenenbaum JB, Kemp C, Griffiths TL, Goodman ND. How to Grow a Mind:
%Statistics, Structure, and Abstraction. Science 2011
%Mar;331(6022):1279--1285. \cite{Tenenbaum2011}
%
%2. Feinerman O, Rotem A, Moses E. Reliable neuronal logic devices from
%patterned hippocampal cultures. Nature Physics 2008 Oct;4(12):967--973.
%\cite{Feinerman2008}
%
%3. Ward SA, Halstead RH. Computation Structures. The MIT Press; 1989.
%\cite{Ward1989}
%
%
%4. Abelson H, Sussman GJ. Structure and Interpretation of Computer
%Programs - 2nd Edition (MIT Electrical Engineering and Computer
%Science). The MIT Press; 1996. \cite{Abelson1996}
%
%5. Barendregt HP. The Lambda Calculus, Its Syntax and Semantics. North
%Holland; 1985. \cite{Barendregt1985}
%
%6. Cummings DD, Wilcox KS, Dichter MA. Calcium-Dependent Paired-Pulse
%Facilitation of Miniature EPSC Frequency Accompanies Depression of EPSCs
%at Hoppocampal Synapes in Culture. The Journal of Neuroscience 1996
%Sept;16(17):5312-5323 \cite{Cummings1996}
%
%7. McClelland JL, Botvinick MM, Noelle DC, Plaut DC, Rogers TT,
%Seidenberg MS, Smith LB. Letting structure emerge: connectionist and
%dynamical systems approaches to cognition. Trends in cognitive sciences
%2010 Aug;14(8):348--56. \cite{McClelland2010}
%
%8. Griffiths TL, Chater N, Kemp C, Perfors A, Tenenbaum JB.
%Probabilistic models of cognition: exploring representations and
%inductive biases. Trends in cognitive sciences 2010 Aug;14(8):357--64. \cite{Griffiths2010}

\end{document}
